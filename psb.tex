%\documentclass[wsdraft]{ws-procs11x85}

\documentclass{ws-procs11x85}
\usepackage{ws-procs-thm}           % comment this line when `amsthm / theorem / ntheorem` package is used

\begin{document}

\title{NSIDES: DRUG EFFECT DISCOVERY USING THE FDA ADVERSE REPORTING SYSTEM}

\author{RAMI S VANGURI}

\address{Department of Biomedical Informatics, Columbia University,\\
New York, NY 10032 USA\\
E-mail: r.vanguri@columbia.edu}

\author{JOSEPH D ROMANO}

\address{Department of Biomedical Informatics, Columbia University,\\
New York, NY 10032 USA\\
E-mail: jdr2160@columbia.edu}

\author{TAL LORBERBAUM}

\address{Department of Biomedical Informatics, Columbia University,\\
New York, NY 10032 USA\\
E-mail: tal.lorberbaum@columbia.edu}

\author{VICTOR NWANKWO}

\address{Department of Biomedical Informatics, Columbia University,\\
New York, NY 10032 USA\\
E-mail: vtn2106@columbia.edu}

\author{CHOONHAN YOUN}

\address{San Diego Supercomputer Center, University of California, San Diego,\\
La Jolla, CA 92093 USA\\
E-mail: cyoun@sdsc.edu}

\author{NICHOLAS P TATONETTI}

\address{Department of Biomedical Informatics, Columbia University,\\
New York, NY 10032 USA\\
E-mail: nick.tatonetti@columbia.edu}

\begin{abstract}
Adverse drug events are a leading cause of morbidity and mortality
around the world. Regulatory agencies, such as the Food and Drug
Administration (FDA), maintain large collections of adverse event
reports, providing an opportunity to retrospectively study drug and
drug combination effects.  We mined the FDA Adverse Event Reporting
System (FAERS) for significant adverse reactions and developed a
database of drug effects, known as nSides. FAERS contains millions of
reports covering thousands of drugs and thousands of effects,
requiring the computing of approximately X billion models. We present
a scalable, distributed, on-demand computational infrastructure which
can be used with spontaneous reporting systems to calculate side
effect significances from drug combinations. Even though nSides is
based on FAERS data, the presented infrastructure is portable can be
applied to any spontaneous reporting system.
\end{abstract}

% required
\bodymatter

\section{Introduction}

Spontaneous reporting systems such as the FDA Adverse Event Reporting
System (FAERS) are important resources for detecting drug adverse
events after a drug is approved (pharmacovigilance). However,
pharmacovigilance algorithms often lead to many false positive and
false negative findings due to issues of confounding, and detection of
drug-drug interactions is an even greater challenge.  We previously
developed databases for off-label drug effects (O\textsc{ffsides}) and
drug interactions (T\textsc{wosides}) using FAERS that account for
these limitations using a novel Statistical Correction for
Uncharacterized Bias (SCRUB)~\cite{Tatonetti2012}.  We re-mined FAERS
with an updated algorithm to populate a new version of the databases,
known as nSides.  nSides also contains a public web gateway
(http://nsides.io/) accessible to researchers, clinicians and patients
alike.

We present the computational infrastructure used to populate the
nSides database.  Additionally, we present an on-demand interface
where users can request drug combination side effect
significances. The same infrastructure and interface can be applied to
any spontaneous reporting system in order to compare drug effects and
bias.


\section{Data Sources}

There are several data sources which are involved in nSides. We use a
curated version of the FDA Adverse Event Reporting System (FAERS)
known as Adverse Event Open Learning through Universal Standardization
(AEOLUS)~\cite{AEOLUS}.  AEOLUS aims to clean and normalize the data
by removing duplicate cases. This is done by applying standardized
vocabularies in the form of RxNorm to map drug names and SNOMED-CT to
map outcomes. The AEOLUS dataset is publicly available.

\section{Methods}

\subsection{Algorithm}
The algorithm used to develop the databases used for nSides is an
updated version of the one used to populate the O\textsc{ffsides} and
T\textsc{wosides} databases.  These databases contain side effect
significances calculated using raw FAERS data~\cite{Tatonetti2012}.
Generally, a standard signal detection algorithm involves conducting a
disproportionality analysis by comparing the observed reporting
frequency of a drug and outcome to the expected reporting frequency of
all other drugs and the outcome. The metric is known as a Proportional
Reporting Ratio (PRR). If the outcome occurred by chance, the
frequencies will be equal and the PRR will be one. If the PRR is much
larger than one, the null hypothesis is rejected. To reduce sampling
variance and selection bias, propensity score matching is implemented
to form the groups used in the disproportionality analysis. This
procedure, known as SCRUB, matches cases and controls between patients
exposed and not exposed to a particular drug (O\textsc{ffsides}) or
two drugs (T\textsc{wosides}) to mitigate confounding biases. Once
cases and controls are matched, the PRR for various side effects are
calculated.

There are several key differences between the O\textsc{ffsides} and
T\textsc{wosides} databases and nSides. The updated algorithm uses a
deep learning model instead of logistic regression to calculate
propensity scores to match cases and controls. As a result, the
computational power required to populate nSides is much
greater. Additionally, nSides is not designed to be limited to
effects of single and interactions of 2 drugs.

tensorflow-gpu

\subsection{Computational Challenge}

Populating the nSides database requires the SCRUB procedure to be run
for each drug or combination of drugs individually to identify
appropriate cases and controls. Once cases and controls are
identified, PRR values need to be calculated over a range of side
effects. Since the AEOLUS dataset contains $\approx$4,000 drugs,
$\approx$5,000,000 reports, and $\approx$8,000 effects to analyze, a
computational challenge emerges.  To deal with this challenge we
employ resources made available by the Open Science Grid (OSG). The
OSG provides access to computing resources for research in the United
States, free of charge. The computing facilities are located at over
100 sites spanning the United States, primarily at universities and
national labs.  The computing infrastructure presented is optimized to
work on the OSG, but can be adapted to work with other grid computing
resources.

The nSides database is initially populated with single drug
effects. To do this, a deep neural network model is generated for each
unique $\approx$4,500 drugs in the AEOLUS dataset. The models are
generated using the TensorFlow machine learning library. Because the
computation involved in deep neural network model generation is more
intensive than the logistic regression models used for the creation of
O\textsc{ffsides} and T\textsc{wosides}, a more complicated
computational infrastructure is used.

\subsection{Computational Structure}
Generating a model for an individual drug involves the following:
\begin{enumerate}
\item Dimensional reduction of the complete AEOLUS dataset: only consider co-reported drugs that appear in at least 1 report
\item Separate exposed and non-exposed reports
\item Generate deep learning models
\item Use scores generated by models to perform propensity score matching
\item Evaluate side effect PRR values
\item Populate nSides database
\end{enumerate}

\subsection{On-demand Interface}

\section{Discussion}


\bibliographystyle{ws-procs11x85}
\bibliography{ws-pro-sample}

\end{document}
