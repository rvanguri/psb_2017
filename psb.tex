%\documentclass[wsdraft]{ws-procs11x85}

\documentclass{ws-procs11x85}
\usepackage{ws-procs-thm}           % comment this line when `amsthm / theorem / ntheorem` package is used
\usepackage{amsmath}

\begin{document}

\title{DRUG EFFECT DISCOVERY USING SPONTANEOUS REPORTING SYSTEMS}

\author{RAMI S VANGURI}

\address{Department of Biomedical Informatics, Columbia University,\\
New York, NY 10032 USA\\
E-mail: r.vanguri@columbia.edu}

\author{JOSEPH D ROMANO}

\address{Department of Biomedical Informatics, Columbia University,\\
New York, NY 10032 USA\\
E-mail: jdr2160@columbia.edu}

\author{TAL LORBERBAUM}

\address{Department of Biomedical Informatics, Columbia University,\\
New York, NY 10032 USA\\
E-mail: tal.lorberbaum@columbia.edu}

\author{VICTOR NWANKWO}

\address{Department of Biomedical Informatics, Columbia University,\\
New York, NY 10032 USA\\
E-mail: vtn2106@columbia.edu}

\author{CHOONHAN YOUN}

\address{San Diego Supercomputer Center, University of California, San Diego,\\
La Jolla, CA 92093 USA\\
E-mail: cyoun@sdsc.edu}

\author{NICHOLAS P TATONETTI}

\address{Department of Biomedical Informatics, Columbia University,\\
New York, NY 10032 USA\\
E-mail: nick.tatonetti@columbia.edu}

\begin{abstract}
Adverse drug events are a leading cause of morbidity and mortality
around the world. Regulatory agencies, such as the Food and Drug
Administration (FDA), maintain large collections of adverse event
reports, providing an opportunity to retrospectively study drug and
drug combination effects.  We mined the FDA Adverse Event Reporting
System (FAERS) for significant adverse reactions and developed a
database of drug effects, known as nSides. FAERS contains millions of
reports covering thousands of drugs and thousands of effects,
requiring the computing of approximately X billion models. We present
a scalable, distributed, on-demand computational infrastructure which
can be used with spontaneous reporting systems to calculate side
effect significances from drug combinations. Even though nSides is
based on FAERS data, the presented infrastructure is portable can be
applied to any spontaneous reporting system.
\end{abstract}

% required
\bodymatter

\section{Introduction}

Spontaneous reporting systems such as the FDA Adverse Event Reporting
System (FAERS) and EudraVigilance are important resources for
detecting drug adverse events during and after the drug approval
process (pharmacovigilance). However, pharmacovigilance algorithms
often lead to many false positive and false negative findings due to
issues of confounding, and detection of drug-drug interactions is an
even greater challenge.  We previously developed databases for
off-label drug effects (O\textsc{ffsides}) and drug interactions
(T\textsc{wosides}) using FAERS that account for these limitations
using a novel Statistical Correction for Uncharacterized Bias
(SCRUB)~\cite{Tatonetti2012}.  We re-mined FAERS with an updated
algorithm to populate a new version of the databases, known as nSides.
nSides also contains a front-end component consisting of a web gateway
(http://nsides.io/) accessible to researchers, clinicians and patients
alike.

We present the computational infrastructure used to populate the
nSides back-end database.  Additionally, we present middleware used to
communicate between users and the back-end database population.  The
communication is done via an on-demand interface where users can
request the nSides back-end database be populated with a specific drug
combination. Drug combinations must be calculated on an on-demand
basis due to the very high number of drug combinations possible in
spontaneous reporting systems. The goal is to populate the back-end
database within 24 hours of a user request.

A key feature of the computing infrastructure presented is portability
to other spontaneous reporting systems, such as EudraVigilance.  This
allows researchers to create their own version of the back-end
database for any use, including incorporation in the nSides web
gateway.  Additionally, using the consistent algorithms across
spontaneous reporting systems allows researchers to evaluate drug
effect differences and bias due to drug approval process differences.

The computational infrastructure with instructions is located on GitHub: https://github.com/tatonetti-lab/nsides


\section{Data Sources}

There are several data sources which are involved in nSides. We use a
curated version of the FDA Adverse Event Reporting System (FAERS)
known as Adverse Event Open Learning through Universal Standardization
(AEOLUS)~\cite{AEOLUS}.  AEOLUS aims to clean and normalize the data
by removing duplicate cases. This is done by applying standardized
vocabularies in the form of RxNorm to map drug names and SNOMED-CT to
map outcomes. The AEOLUS dataset is publicly available.

By using standard vocabularies for drug names and outcomes allows
other spontaneous reporting systems to be adapted to a consistent
format used on the AEOLUS dataset.

\section{Methods}

\subsection{Algorithm}
The algorithm used to develop the databases used for nSides is an
updated version of the one used to populate the O\textsc{ffsides} and
T\textsc{wosides} databases.  These databases contain side effect
significances calculated using raw FAERS data~\cite{Tatonetti2012}.
Generally, a standard signal detection algorithm involves conducting a
disproportionality analysis by comparing the observed reporting
frequency of a drug and outcome to the expected reporting frequency of
all other drugs and the outcome. The metric is known as a Proportional
Reporting Ratio (PRR). If the outcome occurred by chance, the
frequencies will be equal and the PRR will be one. If the PRR is much
larger than one, the null hypothesis is rejected. To reduce sampling
variance and selection bias, propensity score matching is implemented
to form the groups used in the disproportionality analysis. This
procedure, known as SCRUB, matches cases and controls between patients
exposed and not exposed to a particular drug (O\textsc{ffsides}) or
two drugs (T\textsc{wosides}) to mitigate confounding biases. Once
cases and controls are matched, the PRR for various side effects are
calculated.

There are several key differences between the O\textsc{ffsides} and
T\textsc{wosides} databases and nSides. The updated algorithm uses a
deep learning model as well as a logistic regression to calculate
propensity scores to match cases and controls. By using two algorithms
instead of just one and the increased complexity of deep learning
models, the computational power required to populate the nSides
back-end database is much greater. Since nSides is not designed to be
limited to effects of single and interactions of 2 drugs, a much
larger computational infrastructure is required.

\subsection{Computational Challenge}

Populating the nSides database requires the SCRUB procedure to be run
for each drug or combination of drugs individually to identify
appropriate cases and controls. Once cases and controls are
identified, PRR values need to be calculated over a range of side
effects. Since the AEOLUS dataset contains $\approx$4,000 drugs,
$\approx$5,000,000 reports, and $\approx$8,000 effects to analyze, a
computational challenge emerges.  To deal with this challenge we
employ resources made available by the Open Science Grid (OSG) and
Columbia University's computing cluster, Habanero. The OSG provides
access to computing resources for research in the United States, free
of charge. The computing facilities are located at over 100 sites
spanning the United States, primarily at universities and national
labs.  The computing infrastructure presented is optimized to work on
the OSG to increase the portability of the infrastructure to other
spontaneous reporting systems.

The nSides back-end database is initially populated with single drug
effects. To do this, a deep neural network model and a logistic
regression model are generated for each the unique $\approx$4,500
drugs in the AEOLUS dataset. The models are generated using the
TensorFlow and SciKit Learn machine learning libraries,
respectively. The computation involved in deep neural network model
generation is more intensive than solely using logistic regression
models, such as in the creation of O\textsc{ffsides} and
T\textsc{wosides}. As a result, a more complicated computational
infrastructure is used.

Populating the back-end nSides database for drug interactions quickly
becomes intractable.  Shown in Figure~\ref{fig:drug_interactions} are
the number of possible models to compute for the interaction of two
and three drugs.

TODO: Figure with number of drugs (x-axis) and number of models that need to be computed (y-axis)

\subsubsection{Distributed Computing Strategy}
The combinatorial complexity of running all jobs scales on the order
of $\Theta(N^2)$ in the number of drugs. Given that a single job
running on an OSG node takes 4-10 hours to complete, a distributing
computing strategy is necessary to make the total runtime
practical. We utilized two distributed computing systems to determine
the PRR between all pairs of coreported drugs, as described below.

The first of these, as mentioned above, was provided by the OSG. The
OSG uses the HTCondor job submission
software~\cite{beowulfbook-condor}, which handles allocation of
computing resources to jobs submitted by the user. The other
distributed computing system we utilized was Columbia University's
scientific computing cluster, named Habanero. Habanero, like the OSG,
uses a job submission management system, but unlike the OSG, Habanero
uses the Slurm workload manager~\cite{slurm} for user-submitted
jobs. We used both HTCondor and Slurm to submit jobs in a directed
acyclic graph (DAG) configuration, supported using native extensions
to HTCondor and Slurm~\cite{dagman}. In short, the DAG strategy allows
users to take advantage of the fact that many distributed computing
workflows consist of jobs containing common elements, such as initial
data preparation. In a simplified example, a workflow may consist of
one invariant data preparation stage followed by two sequential
machine learning models, where each step relies on the previous step,
and the machine learning models accept variable parameters. Here, the
DAG capabilities of HTCondor and Slurm will only run the first stage
one time, and sequentially pass the results of each stage to the next
stage with the appropriate parameters as the results of the previous
stage are available. This approach allows us to substantially reduce
the combinatorial complexity of running all jobs to a manageable
level.

Since HTCondor and SLURM are the two most common job schedulers, we
release code such that the nSides back-end infrastructure can be
easily deployed on either. This is done to increase the portability of
the infrastructure to other spontaneous reporting systems.

\subsection{Computational Structure}
Generating a model for an individual drug or drug combination involves the following:

\begin{enumerate}
\item Data preparation: Dimensional reduction of the complete AEOLUS dataset to reduce computational complexity of generating models. To do this, we only consider co-reported drugs that appear in at least 1 report
\item Model generation: Generate 20 deep learning and logistic regression models per drug using different subsets of exposed and non-exposed reports. This is done to increase generality of the model.
\item Model evaluation: Use scores generated by models to perform propensity score matching, use propensity score matched cases and controls to calculate side effect PRR values.
\item Populate nSides back-end MongoDB database.
\end{enumerate}

Steps (1) through (3) are performed in a grid computing environment
and (4) is dependent on the hosting structure of the mongoDB database
back-end.

TODO: DAG Figure to be referenced in Distributed Computing Strategy
and Computational Structure





\subsection{On-demand Interface}
The original testing of our approach was performed manually on the
distributed computing servers using a command line interface and shell
scripts. In order to improve the ease of job submission in the future,
we have developed a robust searchable shared-usage gateway to the OSG
resources we have described previously. The gateway uses a
three-tiered architecture consisting of browser-based user interfaces
on the frontend, the OSG job submission system on the backend, and
middleware to facilitate communication between the other two
components. The user interfaces are deployed using the Python Flask
framework, and they access a variety of web services that constitute
the middle tier of the gateway. These web services are arranged in a
way that allows a heterogeneous collection of resources to be accessed
remotely in a uniform fashion.

Additionally, the user interface frontend is bundled alongside a
RESTful Web API (Application Programming Interface) that allows
authenticated users to submit jobs programmatically. This API is
implemented using the Agave tenant service~\cite{dooley2012agave}---a
cloud-based API system designed for developing APIs to be used for
scientific computing. The Agave job API manages all aspects of job
execution and management, including data staging, job submission, job
monitoring, output archiving, event logging, sharing, and
notifications.

\section{Discussion}

Creating an open-source computing infrastructure for mining and
presenting side effect significances from spontaneous reporting
systems has great potential for improving pharmacovigilance. By using
the same algorithm across many systems, it can be possible to evaluate
drug effect significance differences which can be used to reduce the
potential of adverse drug effects.

Due to dataset restrictions, it is not always possible to incorporate
spontaneous reporting system data other than FAERS directly.  For
example, EudraVigilance data is only available to academic
institutions within the European Union. With the infrastructure
presented, it is possible for researchers with access to the
EudraVigilance to form a separate version of nSides which can be
compared with the version made with FAERS data.




\bibliographystyle{ws-procs11x85}
\bibliography{ws-pro-sample}

\end{document}
