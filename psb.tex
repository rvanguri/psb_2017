%\documentclass[wsdraft]{ws-procs11x85}

\documentclass{ws-procs11x85}
\usepackage{ws-procs-thm}           % comment this line when `amsthm / theorem / ntheorem` package is used

\begin{document}

\title{NSIDES: DRUG EFFECT DISCOVERY USING THE FDA ADVERSE REPORTING SYSTEM}

\author{RAMI S VANGURI}

\address{Department of Biomedical Informatics, Columbia University,\\
New York, NY 10032 USA\\
E-mail: r.vanguri@columbia.edu}

\author{JOSEPH D ROMANO}

\address{Department of Biomedical Informatics, Columbia University,\\
New York, NY 10032 USA\\
E-mail: r.vanguri@columbia.edu}

\author{TAL LORBERBAUM}

\address{Department of Biomedical Informatics, Columbia University,\\
New York, NY 10032 USA\\
E-mail: r.vanguri@columbia.edu}

\author{VICTOR NWANKWO}

\address{Department of Biomedical Informatics, Columbia University,\\
New York, NY 10032 USA\\
E-mail: r.vanguri@columbia.edu}

\author{CHOONHAN YOUN}

\address{San Diego Supercomputer Center, University of California, San Diego,\\
La Jolla, CA 92093 USA\\
E-mail: cyoun@sdsc.edu}

\author{NICHOLAS P TATONETTI}

\address{Department of Biomedical Informatics, Columbia University,\\
New York, NY 10032 USA\\
E-mail: nick.tatonetti@columbia.edu}

\begin{abstract}
Adverse drug events are a leading cause of morbidity and mortality
around the world. Regulatory agencies, such as the Food and Drug
Administration (FDA), maintain large collections of adverse event
reports.  These spontaneous reporting systems provide an opportunity
to retrospectively study drug effects from patient population
data. Using an updated version of our previously developed algorithm
to conduct high dimensional propensity score matching, we use the FDA
Adverse Event Reporting System (FAERS) to develop a a database drug
effects, known as nSides. FAERS contains millions of reports,
thousands of drugs and thousands of effects.  In order to calculate
side effect significances over the complete set of drugs reported in
FAERS, we present a computational architecture which is distributed.
In addition, we present an on-demand interface for the members of the
public to request side effect significances for drug interactions.
\end{abstract}

% required
\bodymatter

\section{Introduction}

Spontaneous reporting systems such as the FDA Adverse Event Reporting
System (FAERS) are important resources for detecting drug adverse
events after a drug is approved (pharmacovigilance). However,
pharmacovigilance algorithms often lead to many false positive and
false negative findings due to issues of confounding, and detection of
drug-drug interactions is an even greater challenge.  We previously
developed databases for off-label drug effects (O\textsc{ffsides}) and
drug interactions (T\textsc{wosides}) that account for these
limitations using a novel Statistical Correction for Uncharacterized
Bias (SCRUB)~\cite{Tatonetti2012}.  These databases are available for
download to the public.  In addition to updates to the algorithm, we
have developed a gateway to access the databases, known as nSides.
nSides aims to make these databases accessible to researchers,
clinicians, and patients alike and contains additional features
related to drug safety.  Since it is not feasible to generate models
for every possible drug combination, we develop a novel model request
system which submits jobs to the Open Science Grid and appends the
results to the databases for future access.

\section{Data Sources}

There are several data sources which are involved in nSides. We use a
curated version of the FDA Adverse Event Reporting System (FAERS)
known as Adverse Event Open Learning through Universal Standardization
(AEOLUS)~\cite{AEOLUS}.  AEOLUS aims to clean and normalize the data
by removing duplicate cases. This is done by applying standardized
vocabularies in the form of RxNorm to map drug names and SNOMED-CT to
map outcomes. The AEOLUS dataset is publicly available.

The algorithm used to develop the databases used for nSides are
similar to the previously developed O\textsc{ffsides} and
T\textsc{wosides} databases using raw FAERS data~\cite{Tatonetti2012}.
A standard signal detection algorithm involves conducting a
disproportionality analysis by comparing the observed reporting
frequency of a drug and outcome to the expected reporting frequency of
all other drugs and the outcome. The metric is known as a Proportional
Reporting Ratio (PRR). If the outcome occurred by chance, the
frequencies will be equal and the PRR will be one. If the PRR is much
larger than one, the null hypothesis is rejected. To reduce sampling
variance and selection bias, propensity score matching is implemented
on the FAERS data to form the groups used in the disproportionality
analysis. This procedure, known as SCRUB, matches cases and controls
between patients exposed and not exposed to a particular drug
(O\textsc{ffsides}) or two drugs (T\textsc{wosides}) to mitigate
confounding biases.

\section{Methods}

\subsection{Computational Challenge}

The AEOLUS dataset contains $\approx$4,500 drugs, $\approx$4,500,000
reports and $\approx$7,500 effects. Since SCRUB is performed to find
appropriate controls for a set of cases on a particular drug, a model
needs to be developed for every unique drug. It is easily seen that
modeling drug interactions becomes intractable quickly, with 20
million possible interactions of 2 drugs.  However, many of the 2 drug
combinations do not have enough reports in the AEOLUS dataset to
perform meaningful case-control matching.


\bibliographystyle{ws-procs11x85}
\bibliography{ws-pro-sample}

\end{document}
