%\documentclass[wsdraft]{ws-procs11x85}

\documentclass{ws-procs11x85}
\usepackage{ws-procs-thm}           % comment this line when `amsthm / theorem / ntheorem` package is used

\begin{document}

\title{NSIDES: DRUG EFFECT DISCOVERY USING THE FDA ADVERSE REPORTING SYSTEM}

\author{RAMI S VANGURI}

\address{Department of Biomedical Informatics, Columbia University,\\
New York, NY 10032 USA\\
E-mail: r.vanguri@columbia.edu}

\author{JOSEPH D ROMANO}

\address{Department of Biomedical Informatics, Columbia University,\\
New York, NY 10032 USA\\
E-mail: jdr2160@columbia.edu}

\author{TAL LORBERBAUM}

\address{Department of Biomedical Informatics, Columbia University,\\
New York, NY 10032 USA\\
E-mail: tal.lorberbaum@columbia.edu}

\author{VICTOR NWANKWO}

\address{Department of Biomedical Informatics, Columbia University,\\
New York, NY 10032 USA\\
E-mail: vtn2106@columbia.edu}

\author{CHOONHAN YOUN}

\address{San Diego Supercomputer Center, University of California, San Diego,\\
La Jolla, CA 92093 USA\\
E-mail: cyoun@sdsc.edu}

\author{NICHOLAS P TATONETTI}

\address{Department of Biomedical Informatics, Columbia University,\\
New York, NY 10032 USA\\
E-mail: nick.tatonetti@columbia.edu}

\begin{abstract}
Adverse drug events are a leading cause of morbidity and mortality
around the world. Regulatory agencies, such as the Food and Drug
Administration (FDA), maintain large collections of adverse event
reports, providing an opportunity to retrospectively study drug and
drug combination effects.  We mined the FDA Adverse Event Reporting
System (FAERS) for significant adverse reactions and developed a
database of drug effects, known as nSides. FAERS contains millions of
reports covering thousands of drugs and thousands of effects,
requiring the computing of approximately X billion models.  In order
to calculate side effect significances for all drug combinations
reported in FAERS, we present a distributed, on-demand computational
architecture.
\end{abstract}

% required
\bodymatter

\section{Introduction}

Spontaneous reporting systems such as the FDA Adverse Event Reporting
System (FAERS) are important resources for detecting drug adverse
events after a drug is approved (pharmacovigilance). However,
pharmacovigilance algorithms often lead to many false positive and
false negative findings due to issues of confounding, and detection of
drug-drug interactions is an even greater challenge.  We previously
developed databases for off-label drug effects (O\textsc{ffsides}) and
drug interactions (T\textsc{wosides}) that account for these
limitations using a novel Statistical Correction for Uncharacterized
Bias (SCRUB)~\cite{Tatonetti2012}.  We re-mined FAERS with an updated
algorithm to populate a new version of the databases, known as nSides.
nSides also contains a public web gateway (http://nsides.io/)
accessible to researchers, clinicians and patients alike.


\section{Data Sources}

There are several data sources which are involved in nSides. We use a
curated version of the FDA Adverse Event Reporting System (FAERS)
known as Adverse Event Open Learning through Universal Standardization
(AEOLUS)~\cite{AEOLUS}.  AEOLUS aims to clean and normalize the data
by removing duplicate cases. This is done by applying standardized
vocabularies in the form of RxNorm to map drug names and SNOMED-CT to
map outcomes. The AEOLUS dataset is publicly available.

\section{Methods}

\subsection{Algorithm}
The algorithm used to develop the databases used for nSides is an
updated version of the one used to populate the O\textsc{ffsides} and
T\textsc{wosides} databases.  These databases contain side effect
significances calculated using raw FAERS data~\cite{Tatonetti2012}.
Generally, a standard signal detection algorithm involves conducting a
disproportionality analysis by comparing the observed reporting
frequency of a drug and outcome to the expected reporting frequency of
all other drugs and the outcome. The metric is known as a Proportional
Reporting Ratio (PRR). If the outcome occurred by chance, the
frequencies will be equal and the PRR will be one. If the PRR is much
larger than one, the null hypothesis is rejected. To reduce sampling
variance and selection bias, propensity score matching is implemented
to form the groups used in the disproportionality analysis. This
procedure, known as SCRUB, matches cases and controls between patients
exposed and not exposed to a particular drug (O\textsc{ffsides}) or
two drugs (T\textsc{wosides}) to mitigate confounding biases.

There are several key differences between the O\textsc{ffsides} and
T\textsc{wosides} databases and nSides. The updated algorithm uses a
deep learning model instead of logistic regression to calculate
propensity scores to match cases and controls.

\subsection{Computational Challenge}

The AEOLUS dataset contains $\approx$4,500 drugs, $\approx$4,500,000
reports and $\approx$7,500 effects. In order to populate nSides, a
unique model needs to be created for each drug present in AEOLUS,
presenting a computational challenge.  This becomes even more
intractable when considering drug interactions.



\bibliographystyle{ws-procs11x85}
\bibliography{ws-pro-sample}

\end{document}
